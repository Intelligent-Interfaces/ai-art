\vspace{-.25cm}
\frenchspacing
\begin{abstract}
\noindent From Pablo Picasso to Vincent van Gogh, and beyond, these master artists have displayed depths of the human imagination. In doing so, these artists have inspired and continue to inspire generations of artists and associated artistic movements. Indeed, fine arts has been critical to the development of aesthetics, and in general, human creative expression. The evolution of human aesthetics and culture continues as humanity exchanges the paintbrush and easel for mouse, keyboard, and screen. With this paper, we follow the artistic evolutionary trajectory by applying state-of-the-art machine learning systems for the generation of artistically styled photos. In particular, we review the popular optimization technique neural style transfer (NST) and the generative reinforcement learning framework, generative adversarial networks (GANs). From there, we design and asses the performance of NST-based art experiments using single, dual, and trio GAN models. As a principle result, we outline the mathematical and computational architectures associated with these machine learning systems, as well as their trade-offs as it's related to the generation of artistically styled images. 
\end{abstract}
\vspace{1.0cm}