\vspace{-.25cm}
\frenchspacing
\begin{abstract}
\noindent Since the early renaissance drawing machines have been of interest to artists, designers, inventors, and mathematicians. These drawing machine systems have allowed humanity to explore ideas on assistive tooling and generative art. Such ideas have been distilled through geometry and physics; practically resolving themselves into a diversity of complex autonomous and semi-autonomous machines. In this paper, we unveil the complex history of such mechanisms, highlighting the mathematical and manufacturing principles that make these machines possible. We then introduce the role of computation in modern drafting tools, showcasing how computer science has changed the field of generative art and design. Finally, we propose a novel drawing machine architecture that uses generative adversarial networks (GANs) and associated machine vision and computational geometry algorithms. Using this novel architecture, we discuss the role of such GAN-based drawing systems in industrial robotics and assistive human-inspired artistic expression.
\end{abstract}
\vspace{1.0cm}